\hypertarget{main_LICENSE}{}\section{License and Disclaimer}\label{main_LICENSE}
SQRKal is distributed under the following license, which is included in the source code distribution. It is reproduced in the manual in case you got the library from another source.

 
\begin{quote}
Copyright (c) 2001-2005 Cisco Systems, Inc.  All rights reserved.

Redistribution and use in source and binary forms, with or without
modification, are permitted provided that the following conditions
are met:
\begin{itemize}
\item  Redistributions of source code must retain the above copyright
  notice, this list of conditions and the following disclaimer.
\item Redistributions in binary form must reproduce the above
  copyright notice, this list of conditions and the following
  disclaimer in the documentation and/or other materials provided
  with the distribution.
\item Neither the name of the Cisco Systems, Inc. nor the names of its
  contributors may be used to endorse or promote products derived
  from this software without specific prior written permission.
\end{itemize}
THIS SOFTWARE IS PROVIDED BY THE COPYRIGHT HOLDERS AND CONTRIBUTORS
"AS IS" AND ANY EXPRESS OR IMPLIED WARRANTIES, INCLUDING, BUT NOT
LIMITED TO, THE IMPLIED WARRANTIES OF MERCHANTABILITY AND FITNESS
FOR A PARTICULAR PURPOSE ARE DISCLAIMED. IN NO EVENT SHALL THE
COPYRIGHT HOLDERS OR CONTRIBUTORS BE LIABLE FOR ANY DIRECT,
INDIRECT, INCIDENTAL, SPECIAL, EXEMPLARY, OR CONSEQUENTIAL DAMAGES
(INCLUDING, BUT NOT LIMITED TO, PROCUREMENT OF SUBSTITUTE GOODS OR
SERVICES; LOSS OF USE, DATA, OR PROFITS; OR BUSINESS INTERRUPTION)
HOWEVER CAUSED AND ON ANY THEORY OF LIABILITY, WHETHER IN CONTRACT,
STRICT LIABILITY, OR TORT (INCLUDING NEGLIGENCE OR OTHERWISE)
ARISING IN ANY WAY OUT OF THE USE OF THIS SOFTWARE, EVEN IF ADVISED
OF THE POSSIBILITY OF SUCH DAMAGE.
\end{quote}
\hypertarget{main_Features}{}\section{Supported Features}\label{main_Features}
\hypertarget{main_Installing}{}\section{Installing and Building libSRTP}\label{main_Installing}
 

To install libSRTP, download the latest release of the distribution
from \texttt{srtp.sourceforge.net}.  The format of the names of the
distributions are \texttt{srtp-A.B.C.tgz}, where \texttt{A} is the
version number, \texttt{B} is the major release number, \texttt{C} is
the minor release number, and \texttt{tgz} is the file
extension\footnote{The extension \texttt{.tgz} is identical to
\texttt{tar.gz}, and indicates a compressed tar file.}  You probably
want to get the most recent release.  Unpack the distribution and
extract the source files; the directory into which the soruce files
will go is named \texttt{srtp}.

libSRTP uses the GNU \texttt{autoconf} and \texttt{make}
utilities\footnote{BSD make will not work; if both versions of make
are on your platform, you can invoke GNU make as \texttt{gmake}.}.  In
the \texttt{srtp} directory, run the configure script and then make:
\begin{verbatim}
  ./configure [ options ]       
  make                          
\end{verbatim}
The configure script accepts the following options:
\begin{quote}
\begin{description}
\item[--help]              provides a usage summary.
\item[--disable-debug]     compiles libSRTP without the runtime 
			   dynamic debugging system.
\item[--enable-generic-aesicm] compile in changes for ismacryp
\item[--enable-syslog]     use syslog for error reporting.
\item[--disable-stdout]    diables stdout for error reporting.
\item[--enable-console]    use \texttt{/dev/console} for error reporting
\item[--gdoi]              use GDOI key management (disabled at present).
\end{description}
\end{quote}

By default, dynamic debbuging is enabled and stdout is used for
debugging.  You can use the configure options to have the debugging
output sent to syslog or the system console.  Alternatively, you can
define ERR\_REPORTING\_FILE in \texttt{include/conf.h} to be any other
file that can be opened by libSRTP, and debug messages will be sent to
it.

This package has been tested on the following platforms: Mac OS X
(powerpc-apple-darwin1.4), Cygwin (i686-pc-cygwin), Solaris
(sparc-sun-solaris2.6), RedHat Linux 7.1 and 9 (i686-pc-linux), and
OpenBSD (sparc-unknown-openbsd2.7).


\hypertarget{main_Applications}{}\section{Applications}\label{main_Applications}
 

Several test drivers and a simple and portable srtp application are
included in the \texttt{test/} subdirectory.

\begin{center}
\begin{tabular}{ll}
\hline
Test driver    	& Function tested	\\
\hline
kernel\_driver   & crypto kernel (ciphers, auth funcs, rng) \\
srtp\_driver	& srtp in-memory tests (does not use the network) \\
rdbx\_driver	& rdbx (extended replay database) \\
roc\_driver	& extended sequence number functions \\ 
replay\_driver	& replay database  \\
cipher\_driver	& ciphers  \\
auth\_driver	& hash functions \\
\hline
\end{tabular}
\end{center}

The app rtpw is a simple rtp application which reads words from
/usr/dict/words and then sends them out one at a time using [s]rtp.
Manual srtp keying uses the -k option; automated key management
using gdoi will be added later.

The usage for rtpw is

\texttt{rtpw [[-d $<$debug$>$]* [-k $<$key$>$ [-a][-e]] [-s | -r] dest\_ip
dest\_port][-l]}

Either the -s (sender) or -r (receiver) option must be chosen.  The
values dest\_ip, dest\_port are the IP address and UDP port to which
the dictionary will be sent, respectively.  The options are:
\begin{center}
\begin{tabular}{ll}
  -s		& (S)RTP sender - causes app to send words \\
  -r		& (S)RTP receive - causes app to receve words \\
  -k $<$key$>$      & use SRTP master key $<$key$>$, where the 
		key is a hexadecimal value (without the
                leading "0x") \\
  -e            & encrypt/decrypt (for data confidentiality)
                (requires use of -k option as well)\\
  -a            & message authentication 
                (requires use of -k option as well) \\
  -l            & list the avaliable debug modules \\
  -d $<$debug$>$    & turn on debugging for module $<$debug$>$ \\
\end{tabular}
\end{center}

In order to get a random 30-byte value for use as a key/salt pair, you
can use the \texttt{rand\_gen} utility in the \texttt{test/}
subdirectory.

An example of an SRTP session using two rtpw programs follows:

\begin{verbatim}
[sh1] set k=`test/rand_gen -n 30`
[sh1] echo $k
c1eec3717da76195bb878578790af71c4ee9f859e197a414a78d5abc7451
[sh1]$ test/rtpw -s -k $k -ea 0.0.0.0 9999 
Security services: confidentiality message authentication
set master key/salt to C1EEC3717DA76195BB878578790AF71C/4EE9F859E197A414A78D5ABC7451
setting SSRC to 2078917053
sending word: A
sending word: a
sending word: aa
sending word: aal
sending word: aalii
sending word: aam
sending word: Aani
sending word: aardvark
...

[sh2] set k=c1eec3717da76195bb878578790af71c4ee9f859e197a414a78d5abc7451
[sh2]$ test/rtpw -r -k $k -ea 0.0.0.0 9999 
security services: confidentiality message authentication
set master key/salt to C1EEC3717DA76195BB878578790AF71C/4EE9F859E197A414A78D5ABC7451
19 octets received from SSRC 2078917053 word: A
19 octets received from SSRC 2078917053 word: a
20 octets received from SSRC 2078917053 word: aa
21 octets received from SSRC 2078917053 word: aal
...
\end{verbatim}


\hypertarget{main_Review}{}\section{Secure RTP Background}\label{main_Review}
In this section we review SRTP and introduce some terms that are used in libSRTP. An RTP session is defined by a pair of destination transport addresses, that is, a network address plus a pair of UDP ports for RTP and RTCP. RTCP, the RTP control protocol, is used to coordinate between the participants in an RTP session, e.g. to provide feedback from receivers to senders. An {\itshape SRTP\/} {\itshape session\/} is similarly defined; it is just an RTP session for which the SRTP profile is being used. An SRTP session consists of the traffic sent to the SRTP or SRTCP destination transport addresses. Each participant in a session is identified by a synchronization source (SSRC) identifier. Some participants may not send any SRTP traffic; they are called receivers, even though they send out SRTCP traffic, such as receiver reports.

RTP allows multiple sources to send RTP and RTCP traffic during the same session. The synchronization source identifier (SSRC) is used to distinguish these sources. In libSRTP, we call the SRTP and SRTCP traffic from a particular source a {\itshape stream\/}. Each stream has its own SSRC, sequence number, rollover counter, and other data. A particular choice of options, cryptographic mechanisms, and keys is called a {\itshape policy\/}. Each stream within a session can have a distinct policy applied to it. A session policy is a collection of stream policies.

A single policy can be used for all of the streams in a given session, though the case in which a single {\itshape key\/} is shared across multiple streams requires care. When key sharing is used, the SSRC values that identify the streams {\bfseries must} be distinct. This requirement can be enforced by using the convention that each SRTP and SRTCP key is used for encryption by only a single sender. In other words, the key is shared only across streams that originate from a particular device (of course, other SRTP participants will need to use the key for decryption). libSRTP supports this enforcement by detecting the case in which a key is used for both inbound and outbound data.\hypertarget{main_Overview}{}\section{libSRTP Overview}\label{main_Overview}
libSRTP provides functions for protecting RTP and RTCP. RTP packets can be encrypted and authenticated (using the srtp\_\-protect() function), turning them into SRTP packets. Similarly, SRTP packets can be decrypted and have their authentication verified (using the srtp\_\-unprotect() function), turning them into RTP packets. Similar functions apply security to RTCP packets.

The typedef srtp\_\-stream\_\-t points to a structure holding all of the state associated with an SRTP stream, including the keys and parameters for cipher and message authentication functions and the anti-\/replay data. A particular srtp\_\-stream\_\-t holds the information needed to protect a particular RTP and RTCP stream. This datatype is intentionally opaque in order to better seperate the libSRTP API from its implementation.

Within an SRTP session, there can be multiple streams, each originating from a particular sender. Each source uses a distinct stream context to protect the RTP and RTCP stream that it is originating. The typedef srtp\_\-t points to a structure holding all of the state associated with an SRTP session. There can be multiple stream contexts associated with a single srtp\_\-t. A stream context cannot exist indepent from an srtp\_\-t, though of course an srtp\_\-t can be created that contains only a single stream context. A device participating in an SRTP session must have a stream context for each source in that session, so that it can process the data that it receives from each sender.

In libSRTP, a session is created using the function srtp\_\-create(). The policy to be implemented in the session is passed into this function as an srtp\_\-policy\_\-t structure. A single one of these structures describes the policy of a single stream. These structures can also be linked together to form an entire session policy. A linked list of srtp\_\-policy\_\-t structures is equivalent to a session policy. In such a policy, we refer to a single srtp\_\-policy\_\-t as an {\itshape element\/}.

An srtp\_\-policy\_\-t strucutre contains two crypto\_\-policy\_\-t structures that describe the cryptograhic policies for RTP and RTCP, as well as the SRTP master key and the SSRC value. The SSRC describes what to protect (e.g. which stream), and the crypto\_\-policy\_\-t structures describe how to protect it. The key is contained in a policy element because it simplifies the interface to the library. In many cases, it is desirable to use the same cryptographic policies across all of the streams in a session, but to use a distinct key for each stream. A crypto\_\-policy\_\-t structure can be initialized by using either the crypto\_\-policy\_\-set\_\-rtp\_\-default() or crypto\_\-policy\_\-set\_\-rtcp\_\-default() functions, which set a crypto policy structure to the default policies for RTP and RTCP protection, respectively.\hypertarget{main_Example}{}\section{Example Code}\label{main_Example}
This section provides a simple example of how to use libSRTP. The example code lacks error checking, but is functional. Here we assume that the value ssrc is already set to describe the SSRC of the stream that we are sending, and that the functions get\_\-rtp\_\-packet() and send\_\-srtp\_\-packet() are available to us. The former puts an RTP packet into the buffer and returns the number of octets written to that buffer. The latter sends the RTP packet in the buffer, given the length as its second argument.

\begin{DoxyVerb}
   srtp_t session;   
   srtp_policy_t policy;
   uint8_t key[30];

   // initialize libSRTP 
   srtp_init();                                  

   // set policy to describe a policy for an SRTP stream 
   crypto_policy_set_rtp_default(&policy.rtp);   
   crypto_policy_set_rtcp_default(&policy.rtcp); 
   policy.ssrc = ssrc;                            
   policy.key  = key;
   policy.next = NULL;

   // set key to random value 
   crypto_get_random(key, 30);          

   // allocate and initialize the SRTP session 
   srtp_create(&session, policy);  
   
   // main loop: get rtp packets, send srtp packets
   while (1) {
      char rtp_buffer[2048];
      unsigned len;

      len = get_rtp_packet(rtp_buffer);
      srtp_protect(session, rtp_buffer, &len);
      send_srtp_packet(rtp_buffer, len);
   }
\end{DoxyVerb}
\hypertarget{main_ISMAcryp}{}\section{ISMA Encryption Support}\label{main_ISMAcryp}
The Internet Streaming Media Alliance (ISMA) specifies a way to pre-\/encrypt a media file prior to streaming. This method is an alternative to SRTP encryption, which is potentially useful when a particular media file will be streamed multiple times. The specification is available online at \href{http://www.isma.tv/specreq.nsf/SpecRequest.}{\tt http://www.isma.tv/specreq.nsf/SpecRequest.}

libSRTP provides the encryption and decryption functions needed for ISMAcryp in the library  libaesicm.a, which is included in the default Makefile target. This library is used by the MPEG4IP project; see \href{http://mpeg4ip.sourceforge.net/.}{\tt http://mpeg4ip.sourceforge.net/.}

Note that ISMAcryp does not provide authentication for RTP nor RTCP, nor confidentiality for RTCP. ISMAcryp RECOMMENDS the use of SRTP message authentication for ISMAcryp streams while using ISMAcryp encryption to protect the media itself. 